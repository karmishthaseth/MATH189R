{\rtf1\ansi\ansicpg1252\cocoartf2511
\cocoatextscaling0\cocoaplatform0{\fonttbl\f0\fswiss\fcharset0 Helvetica;}
{\colortbl;\red255\green255\blue255;}
{\*\expandedcolortbl;;}
\paperw11900\paperh16840\margl1440\margr1440\vieww10800\viewh8400\viewkind0
\pard\tx566\tx1133\tx1700\tx2267\tx2834\tx3401\tx3968\tx4535\tx5102\tx5669\tx6236\tx6803\pardirnatural\partightenfactor0

\f0\fs24 \cf0 \\documentclass\{article\}\
\\usepackage[utf8]\{inputenc\}\
\\usepackage\{amsmath\}\
\\usepackage\{textgreek\}\
\\linespread\{1.5\}\
\
\{\\Huge \\title\{\\huge Home Work 1\}\}\
\\author\{\\large Karmishtha Seth \}\
\\date\{\\large 12 June 2020\}\
\
\\begin\{document\}\
\
\\maketitle\
\
\{\\huge 1\}\
\\bigbreak\
\
\{\\large If A and b are constants, then E[Ax + b] = AE[x] +b\}\
\\bigbreak\
\{\\large To show this:\}\
\\bigbreak\
\{\\large Sum $\\sum ((Ax+b)P(x))= \{E[Ax + b]\}$\}\
\\bigbreak\
\{\\large $A\\cdot ($$\\sum (xP(x)))$ + $b\\cdot ($$\\sum (P(x)))$= \{E[Ax + b]\}\}\
\\bigbreak\
\{\\large Therefore, E[y] = E[Ax + b] = AE[x] + b\}\
\\bigbreak\
\{\\large We know that cov[x] = ($E)\\cdot [(x - E[x])(x - E[x]^T)]$\}\
\\bigbreak\
\{\\large By definition, cov[y] = cov[Ax + b]\} \
\\bigbreak\
\{\\large = $E\\cdot [(Ax + b - E[Ax + b])(Ax + b - E[Ax + b])^T]$\}\
\\bigbreak\
\{\\large = $E\\cdot [(Ax + b - AE[x] - b)(Ax + b - AE[x] - b)^T]$\}\
\\bigbreak\
\{\\large = $E\\cdot [A(x - E[x])(x - E[x]^T A^T]$\}\
\\bigbreak\
\{\\large = $AE\\cdot [(x - E[x])(x - E[x])^T]A^T$\}\
\\bigbreak\
\{\\large = $A\\cdot [cov[x]A^T]$ = $A\\cdot [$$\\sum A^\{T\}$$]$\}\
\\bigbreak\
\\bigbreak\
\{\\huge 2\}\
\\bigbreak\
\{\\large D = \{(x,y)\} = \{(0,1), (2,3), (3,6), (4,8)\}\}\
\\bigbreak\
\{\\large (a).\} \
\\begin\{gather\}\
X = \\begin\{bmatrix\} 1 & 0 \\\\ 1 & 2 \\\\ 1 & 3 \\\\1 & 4 \\end\{bmatrix\}\
\\end\{gather\}\
\\bigbreak\
\\begin\{gather\}\
y = \\begin\{bmatrix\} 1 \\\\ 3 \\\\ 6 \\\\8 \\end\{bmatrix\}\
\\end\{gather\}\
\\bigbreak\
\{\\large from which we can derive:\}\
\\bigbreak\
\\begin\{gather\}\
(X^T)X = \\begin\{bmatrix\} 1 & 1 & 1 & 1\\\\ 0 & 2 & 3 & 4 \\end\{bmatrix\}\\begin\{bmatrix\} 1 & 0 \\\\ 1 & 2 \\\\ 1 & 3 \\\\1 & 4 \\end\{bmatrix\} = \\begin\{bmatrix\} 4 & 9\\\\ 9 & 29 \\end\{bmatrix\}\
\\end\{gather\}\
\{\\large and we also get:\}\
\\begin\{gather\}\
(X^T)y = \\begin\{bmatrix\} 1 & 1 & 1 & 1\\\\ 0 & 2 & 3 & 4 \\end\{bmatrix\}\\begin\{bmatrix\} 1 \\\\ 3 \\\\ 6 \\\\8 \\end\{bmatrix\} = \\begin\{bmatrix\} 18\\\\ 56 \\end\{bmatrix\}\
\\end\{gather\}\
\{\\large By setting the partial derivatives of the least squares equal to zero, we can see that (X^T)X $\\theta^*$. By using Cramer's rule, we get:\}\
\\begin\{gather\}\
\\frac \{\\begin\{vmatrix\} 18 & 9\\\\ 56 & 29 \\end\{vmatrix\}\}\{\\begin\{vmatrix\} 4 & 9\\\\ 9 & 29 \\end\{vmatrix\}\} = \\frac\{18\}\{35\} \
\\end\{gather\}\
\{\\large (5) = $\\theta_0 ^* $ and\}\
\\begin\{gather\}\
\\frac \{\\begin\{vmatrix\} 4 & 18\\\\ 9 & 56 \\end\{vmatrix\}\}\{\\begin\{vmatrix\} 4 & 9\\\\ 9 & 29 \\end\{vmatrix\}\} = \\frac\{62\}\{35\} \
\\end\{gather\}\
\{\\large (6) = $\\theta_1 ^* $. From the above, we can see that y = $\\theta_0$ - $\\theta_0$x.\}\
\\bigbreak\
\{\\large (b).\}\
\{\\large Using normal equation, we get:\}\
\{\\large $\\theta^* = $((X^T X)^\{-1\})\\cdot (X^T y$)$\}\
\\bigbreak\
\\begin\{gather\}\
= \\begin\{bmatrix\} 4 & 9\\\\ 9 & 29 \\end\{bmatrix\}^\{-1\} \\begin\{bmatrix\} 1 & 1 & 1 & 1\\\\ 0 & 2 & 3 & 4 \\end\{bmatrix\} \\begin\{bmatrix\} 1 \\\\ 3 \\\\ 6 \\\\8 \\end\{bmatrix\}\
\\end\{gather\}\
\\bigbreak\
\\begin\{gather\}\
=\\frac\{1\}\{35\}\\begin\{bmatrix\} 29 & -9\\\\ -9 & 4 \\end\{bmatrix\} \\begin\{bmatrix\} 1 & 1 & 1 & 1\\\\ 0 & 2 & 3 & 4 \\end\{bmatrix\} \\begin\{bmatrix\} 1 \\\\ 3 \\\\ 6 \\\\8 \\end\{bmatrix\}\
\\end\{gather\}\
\\bigbreak\
\\begin\{gather\}\
=\\frac\{1\}\{35\}\\begin\{bmatrix\} 29 & 11 & 2 & -7\\\\ -9 & 1 & 3 & 7 \\end\{bmatrix\} \\begin\{bmatrix\} 1 \\\\ 3 \\\\ 6 \\\\8 \\end\{bmatrix\}\
\\end\{gather\}\
\\bigbreak\
\\begin\{gather\}\
= \\frac\{1\}\{35\} \\begin\{bmatrix\} 18\\\\ 62 \\end\{bmatrix\} = \\begin\{bmatrix\} \\frac\{18\}\{35\}\\\\ \\frac\{62\}\{35\} \\end\{bmatrix\}\
\\end\{gather\}\
\{\\large thus we can see that solutions in (a) and (b) are the same\}\
\\bigbreak\
\{\\large (c) & (d). Refer to files in GitHub repository\}\
\\end\{document\}}